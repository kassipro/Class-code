\documentclass{article}
\usepackage[utf8]{inputenc}
\usepackage{amsmath}
\usepackage{amssymb}
\usepackage{hyperref}
\title{Induction Homework}
\date{\today}
\author{CS161}
\begin{document}
\maketitle


\section{General Instructions}
Refer to the PDF in the class code (july-11/resources): \\
\href{https://github.com/csu161/Class-code/blob/master/july-11/resources/inductive-proof-examples.pdf}{Inductive Proof Examples}


\section{Series Summations}
\begin{enumerate}
\item Show that \(1^3+2^3+...+n^3=[n(n+1)/2]^2\) for every positive integer n.
\item Show that \(1*1! + 2*2!+...+n*n! = (n+1)!-1\) for ever positive integer n.
\item Show that \(1^2 + 3^2 + 5^2+...+(2n+1)^2=(n+1)(2n+1)(2n+3)/3\) for ever positive integer n.
\item Show that the sum of the first n even positive integers is \(n(n+1)\).
\end{enumerate}


\section{Problems involving inequality}
\begin{enumerate}
\item Prove that for all integers \(n\leq 4: n^2\leq n!\)
\item Prove that for all integers \(n>1 : n!<n^n\)
\end{enumerate}


\section{Problems involving divisibility}
There is an example in the link above.
\begin{enumerate}
\item Prove that 3 divides \(n^3 + 2n\) for every non-negative integer n.
\item Prove that 6 divides \(n^3-n\) for every non-negative integer n.
\end{enumerate}


\section{Problems from other domains}
\begin{enumerate}
\item Prove that every amount of postage of 6 cents or higher can be formed using just 2 cent and 5 cent stamps.
\end{enumerate}


\end{document}
