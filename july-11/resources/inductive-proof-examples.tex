\documentclass{article}
\usepackage[utf8]{inputenc}
\usepackage{amsmath}
\usepackage{amssymb}
\title{Inductive proof examples}
\begin{document}
\section{Sum from 1 to n}
We want to prove that \(1 + 2 + ... + n \) can be calculated with \( \frac{n(n+1)}{2} \)
\subsection{Definitions}
We will define a function to let us talk about the sum of numbers from 1 to $n$. Let:
\begin{equation}
F(n) = 1 + 2 + ... + n
\end{equation}
We will define a predicate to let us talk about the relationship between $F(n)$ and the shortcut calculation. Let:
\begin{equation}
P(n): F(n) = \frac{n(n + 1)}{2}
\end{equation}
Note that $P(n)$ evaluates to a boolean. It can be true or false for any particular $n$. It is true for a particular value of $n$ if $F(n)$ does in fact equal \( \frac{n(n + 1)}{2} \) and it is false if these two things are not equal.

\subsection{Goal}
Our goal is to prove that $P(n)$ holds (is true) for all values of $n$ greater than 0. Prove:
\begin{equation}
\forall n \in N : P(n)
\end{equation}

\subsection{Proof by induction}
\subsubsection{Base case}
To show our base case $P(1)$ is true, we will state the base case, then show that the left side does in fact equal the right side. Prove:
\begin{equation}
P(1): F(1) = \frac{1(1 + 1)}{2}
\end{equation}
\[ F(1) = 1 \]
\[ \frac{1(1 + 1)}{2} = \frac{2}{2} = 1 \]

\subsubsection{Inductive step}
We will prove that \textbf{if} $P(k)$ holds (is true) for some $k \in N$, \textbf{then} $P(k + 1)$ is also true. Prove:
\begin{equation}
P(k) \implies P(k + 1): \forall k \in N
\end{equation}
We start with the \textit{inductive hypothesis}, we assume for the time that $P(k)$ holds. Assume:
\begin{equation}
P(k): F(k) = \frac{k(k + 1)}{2}
\end{equation}
Now, assuming that $P(k)$ is true, prove:
\begin{equation}
P(k + 1): F(k + 1) = \frac{(k + 1)((k + 1) + 1)}{2}
\end{equation}
By definition:
\[ F(k + 1) = 1 + 2 + ... + k + (k + 1) \]
which is by definition:
\[ F(k + 1) = F(k) + (k + 1) \]
which by our inductive hypothesis is:
\[ F(k + 1) = \frac{k(k + 1)}{2} + (k + 1) \]
simplifying is:
\[ F(k + 1) = (k + 1) (\frac{k}{2} + 1) \]
which is equivalent to:
\[ F(k + 1) = (k + 1) (\frac{k}{2} + \frac{2}{2}) \]
which simplifies to:
\[ F(k + 1) = \frac{(k + 1) (k + 2)}{2} \]
which is clearly:
\[ F(k + 1) = \frac{(k + 1) ((k + 1) + 1)}{2} \]
And so we have proved $P(k + 1)$ (7) by showing that the left side is equal to the right side (assuming that $P(k)$ is true).
\subsection{Conclusion}
We have proved that $P(n)$ holds for a base case of $P(1)$ and that for all $k \in N$, $P(k)$ being true implies that $P(k + 1)$ is also true. Therefore $P(n)$ holds for all $n > 0$ (all natural numbers).
\[P(1): F(1) = \frac{1(1 + 1)}{2}\]
\[P(k) \implies P(k + 1): \forall k \in N \]
\[ \therefore P(n): \forall n \in N \]


\section{Making postage with 3 and 5 cent stamps}
We want to prove that all postage amounts greater than or equal to 8 cents can be made with combinations of 3 and 5 cent stamps


\section{Another summation}
We want to prove that $ 1 + 4 + 7 + ... + (3n - 2) $ can be calculated with $ \frac{n(3n - 1)}{2} $


\section{Proof with inequality}
We want to prove that for any number $n$ greater than or equal to 7, $n!$ is greater than $3^n$.
\[n! > 3^n: n \geq 7 \]


\end{document}
